As of the writing of this report, the drone prototype is not yet capable of stable flight. All essential subsystems and components have been implemented and verified individually. The hardware has been tested and is operational, except for some inconsistencies with the motors, which may require replacement. The firmware architecture and control logic function correctly; however, the closed-loop performance of the \gls{pid} controller has not been validated due to limited testing opportunities. With further refinement and testing, the project goals can be fully achieved.

\subsection*{Recommendations for Achieving Project Goals}

\begin{itemize}
    \item \textbf{Motor Replacement and Testing:} Replace the motors and conduct testing to ensure sufficient thrust and balanced torque. Document results in the Verification and Validation (V\&V) table.
    
    \item \textbf{PID Tuning:} Perform PID tuning under hover conditions once stable thrust is achieved. Adjust gains incrementally and log system responses to optimise stability and control.
    
    \item \textbf{Assembly and Connectivity:} Simplify the mechanical and electrical assembly. Use modular connectors or streamlined wiring layouts to reduce manual work and improve reliability.
    
    \item \textbf{Sensor Upgrade:} Consider replacing the \gls{mpu6050} with a more modern \gls{imu} to improve data accuracy, reduce noise, and enhance stability during flight.
    
    \item \textbf{Future Hardware Improvements:} Evaluate brushless motors, electronic speed controllers (\gls{esc}), and higher-voltage batteries to improve thrust and control precision while maintaining safety with the current frame and software limits.
\end{itemize}

Overall, the system foundation is solid. With incremental hardware replacements, \gls{pid} optimisation, and moderate design refinements, the drone can achieve stable and reliable flight and meet the original project objectives.
