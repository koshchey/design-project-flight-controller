\subsection{Stakeholder Engagement}
Stakeholder engagement was conducted continuously throughout the project to ensure that client expectations and engineering objectives were aligned. Engagement occurred primarily during the first six weeks (Weeks 1–6), which corresponded to the initial requirements elicitation and design clarification phases. Communication was maintained through information sessions, formal technical query (TQ) submissions, and direct email correspondence with both the client and the unit coordinator. During the early project stages, stakeholder engagement focused on understanding and refining the drone’s intended capabilities, constraints, and performance criteria. Key outcomes included clarification of power supply specifications, obstacle detection behaviours, and the structure of autonomous flight patterns. Stakeholder engagement included:

\begin{table}[H]
\centering
\caption{Iterative Engineering Process Followed}
\begin{tabular}{|p{3cm}|p{10cm}|}
\hline
\textbf{Step} & \textbf{Description} \\ \hline
Information Sessions & Weekly sessions with the client to discuss requirements and ask any additional questions. \\ \hline
Technical Queries (TQs) & Formal written submissions to document design uncertainties and receive responses. \\ \hline
Email Correspondence & Direct communication used to clarify requirements due to changes in the project scope. \\ \hline
\end{tabular}
\end{table}

% \subsection{Outcomes of Engagement}

% A detailed record of all stakeholder interactions, including questions, responses, is given below:


\begin{table}[H]
\centering
\caption{Client Meetings, Information Sessions, and Technical Queries - Summary (Weeks 1-5)}
\begin{tabular}{|p{2cm}|p{9.5cm}|p{3cm}|}
\hline
\textbf{Date / Week} & \textbf{Item / Notes} & \textbf{Priority / Additional Notes} \\ \hline
24/07/2025 (W1) & Open-source software must be used. & -- \\ \hline
 & Client requirements may be negotiated with the client. & -- \\ \hline
31/07/2025 (W2) & \textbf{Q1:} Desired action when drone encounters obstacle (stop, reverse, turn around). & Medium \\ \hline
 & \textbf{Client Response:} Stop and hover; shutdown if obstruction persists. Both options desirable. & -- \\ \hline
 & \textbf{Q2:} Requirements for drone movement patterns—preprogrammed or flexible. & High \\ \hline
 & \textbf{Client Response:} Preprogrammed patterns given 1-2 weeks before demo. Custom pattern: simple polygon. & Plan flexible design early \\ \hline
07/08/2025 (W3) & \textbf{Q3:} Two 1S batteries instead of one 2S? Max total capacity 800mAh? & Medium \\ \hline
 & \textbf{Client Response:} Only one battery allowed. Total capacity < 800mAh. & -- \\ \hline
 & Other relevant info: Reaction to wall/obstacles same; automatic obstacle detection and shutdown. & -- \\ \hline
15/08/2025 (W4) & No technical queries this week. & -- \\ \hline
23/08/2025 (W5) & \textbf{Q4:} Mounting IR sensor (VL53L0X) with perpendicular orientation to main PCB. Breakout allowed? & Medium \\ \hline
 & \textbf{Client Response:} Yes. & -- \\ \hline
 & \textbf{Q5:} Can we place coloured/reflective tape for start position? & Low \\ \hline
 & \textbf{Client Response:} Yes. & -- \\ \hline
\end{tabular}
\end{table}

\begin{table}[H]
\centering
\caption{Other Information - Summary (Weeks 1-5)}
\begin{tabular}{|p{2.5cm}|p{12cm}|}
\hline
\textbf{Date / Week} & \textbf{Item / Notes} \\ \hline
24/07/2025 (W1) &
Drifting should be below 10 cm before stabilisation.  \newline 
Object detection is for non-transparent objects only.  \newline 
Surveillance camera must record video and optionally allow hover height adjustment. \newline
Final testing location: MATH151. \newline
ESCs must be part of the PCB. \\ \hline
07/08/2025 (W3) &
No obstacles on ground during demo. \newline
Can use magnetometer and multiple MCU chips, particularly for camera. \newline Camera does not need to be ESP32-based \newline \\ \hline
\end{tabular}
\end{table}

\subsubsection{Requirement Changes} \leavevmode

During Week 6, the project team and client reviewed and refined the ranked and additional requirements.  The proposed changes involved removing requirements that were no longer feasible given the reduced project capacity and the updated system scope, as the number of members had reduced significantly. This was from six members initially to four by August 1, and two by August 25. No new requirements were added; instead, the scope was greatly reduced. The description of these changes can be found in Appendix~\ref{app:req-changes}.

% \pagebreak
\subsubsection{Stakeholder Engagement Summary} \leavevmode

Through structured communication, the team achieved the following: \

\begin{tabular}{|p{4cm}|p{11.5cm}|}
\hline
\rowcolor{gray!15}
\textbf{Activity} & \textbf{Outcome / Actions Closed} \\
\hline
Initial Client Information Sessions (Weeks 1–2) & Defined project expectations and clarified client preferences. Confirmed use of open-source software and flexibility for negotiating requirements. \\ \hline
Technical Query Submissions (Weeks 2–5) & Established operational behaviours for obstacle detection and response (stop, hover, or shutdown). Confirmed flexibility for flight paths with both predefined and custom polygonal trajectories. \\ \hline
Hardware Clarifications & Defined hardware constraints including the use of a single battery (<800mAh), integrated PCB with ESCs, and allowance for breakout components (e.g., IR sensor). \\ \hline
Requirement Tracking and Documentation & Logged all communications and technical clarifications in a shared document. Updated requirements list to reflect client feedback and design feasibility. \\ \hline
Week 6 Client Meeting – Requirement Review & Reviewed all ranked and additional requirements to align project scope with reduced team capacity. Prioritised core functionalities such as flight stability, safety, and manual override. \\ \hline
Requirement Rationalisation & Removed redundant or infeasible requirements (e.g., complex autonomous navigation and multi-pattern path options). Ensured compliance with budget and technical constraints. \\ \hline
Final Client Approval & Confirmed revised and re-ranked requirements list with the client. Finalised scope for design and testing phases based on approved specification. \\ \hline
\end{tabular}