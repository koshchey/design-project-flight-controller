\subsection{Stakeholder Engagement}
Stakeholder engagement was conducted continuously throughout the project to ensure that client expectations and engineering objectives were aligned. Engagement occurred primarily during the first six weeks (Weeks 1–6), which corresponded to the initial requirements elicitation and design clarification phases. Communication was maintained through information sessions, formal technical query (TQ) submissions, and direct email correspondence with both the client and the unit coordinator. During the early project stages, stakeholder engagement focused on understanding and refining the drone’s intended capabilities, constraints, and performance criteria. Key outcomes included clarification of power supply specifications, obstacle detection behaviours, and the structure of autonomous flight patterns. Stakeholder engagement included:

\begin{table}[H]
\centering
\caption{Iterative Engineering Process Followed}
\begin{tabular}{|p{3cm}|p{10cm}|}
\hline
\textbf{Step} & \textbf{Description} \\ \hline
Information Sessions & Weekly sessions with the client to discuss requirements and ask any additional questions. \\ \hline
Technical Queries (TQs) & Formal written submissions to document design uncertainties and receive responses. \\ \hline
Email Correspondence & Direct communication used to clarify requirements due to changes in the project scope. \\ \hline
\end{tabular}
\end{table}

A detailed record of all stakeholder interactions, including questions, responses, is given in Appendix~\ref{app:stakeholder-engagement}.

\subsubsection{Stakeholder Engagement Summary} \leavevmode

Through structured communication, the team achieved the following: \

\begin{tabular}{|p{4cm}|p{11.5cm}|}
\hline
\rowcolor{gray!15}
\textbf{Activity} & \textbf{Outcome / Actions Closed} \\
\hline
Initial Client Information Sessions (Weeks 1–2) & Defined project expectations and clarified client preferences. Confirmed use of open-source software and flexibility for negotiating requirements. \\ \hline
Technical Query Submissions (Weeks 2–5) & Established operational behaviours for obstacle detection and response (stop, hover, or shutdown). Confirmed flexibility for flight paths with both predefined and custom polygonal trajectories. \\ \hline
Hardware Clarifications & Defined hardware constraints including the use of a single battery (<800mAh), integrated PCB with ESCs, and allowance for breakout components (e.g., IR sensor). \\ \hline
Requirement Tracking and Documentation & Logged all communications and technical clarifications in a shared document. Updated requirements list to reflect client feedback and design feasibility. \\ \hline
Week 6 Client Meeting – Requirement Review & Reviewed all ranked and additional requirements to align project scope with reduced team capacity. Prioritised core functionalities such as flight stability, safety, and manual override. \\ \hline
Requirement Rationalisation & Removed redundant or infeasible requirements (e.g., complex autonomous navigation and multi-pattern path options). Ensured compliance with budget and technical constraints. \\ \hline
Final Client Approval & Confirmed revised and re-ranked requirements list with the client. Finalised scope for design and testing phases based on approved specification. \\ \hline
\end{tabular}

\subsubsection{Requirement Changes} \leavevmode

During Week 6, the project team and client reviewed and refined the ranked and additional requirements.  The proposed changes involved removing requirements that were no longer feasible given the reduced project capacity and the updated system scope, as the number of members had reduced significantly. The description of these changes can be found in Appendix~\ref{app:req-changes}.